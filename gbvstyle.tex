%%%%%%%%%%%%%%%%%%%%%%%%%%%%%%%%%%%%%%%%%%%%%%%%%%%%%%%%%%%%%%%%%%%%%%%%%%%%%%%%
% GBV/VZG Style für A4-Dokumente
% Anpassungen zum Pandoc-Standard-Stylesheet
% 
% -V documentclass=scrreprt --include-in-header=gbvstyle.tex
%%%%%%%%%%%%%%%%%%%%%%%%%%%%%%%%%%%%%%%%%%%%%%%%%%%%%%%%%%%%%%%%%%%%%%%%%%%%%%%%

% Benötigte Packages
\usepackage{tikz}
\usepackage{scrpage2}
\usepackage{lastpage}

% GBV-Farbe (Blau)
\definecolor{GBV}{RGB}{89,124,199}

% TODO: VZG Logo
% \pgfdeclareimage[height=10mm]{VZGLogo}{VZGLogo.eps}

% Seitenlayout
\KOMAoptions{twoside}
\usepackage[a4paper,margin=2.5cm,bottom=2.5cm,top=2.5cm]{geometry}

% Serifenlose Schrift
\renewcommand{\familydefault}{\sfdefault}

% TODO: Schriftart Calibiri verwenden

\makeatletter

% Kapitelüberschriften
\renewcommand*\chapterheadstartvskip{\vspace*{-\topskip}}
\renewcommand*{\chapterheadendvskip}{\vspace*{0.5ex}}
\renewcommand*{\chapterpagestyle}{scrheadings}

% Abstand vor und nach Überschriften (siehee http://www.komascript.de/node/1299):
% Achtung: Definition hängt möglicherweise von KomaScript version ab!
\renewcommand\section{\@startsection{section}{1}{\z@}%
  {0.75ex \@plus 0.25ex \@minus .25ex}%
  {0.5ex \@plus 0.125ex \@minus .125ex}%
  {%\ifnum \scr@compatibility>\@nameuse{scr@v@2.96}\relax
    %\setlength{\parfillskip}{\z@ plus 1fil}\fi%
    \raggedsection\normalfont\sectfont\nobreak\size@section}%
}
% TODO: \abovecaptionskip and \belowcaptionskip für Bilder

\setkomafont{chapter}{\fontsize{16bp}{14bp}\selectfont} 
\setkomafont{section}{\fontsize{14bp}{12bp}\selectfont} 

\addtokomafont{sectioning}{\color{GBV}} % chapter, section...
\addtokomafont{caption}{\color{GBV}}
\addtokomafont{captionlabel}{\color{GBV}}
\addtokomafont{chapterentry}{\color{GBV}}
\addtokomafont{chapterentrypagenumber}{\color{GBV}}

\tikzset{page position/.style={remember picture,overlay,shift={(#1)}}}

% Kopf- und Fußzeile
\pagestyle{scrheadings}
\clearscrheadfoot

\lohead{% Kopfzeile linke Seite
  \tikz[page position=current page.north west]
  \draw [fill,GBV] (0mm,-6.4mm) rectangle +(\paperwidth,-11mm);
  \tikz[page position=current page.north west]
  \node [anchor=west,text=white] at (25mm,-11.9mm) { \upshape\textsf{\@date} };
}
\rehead{% Kopfzeile rechte Seite
  \tikz[page position=current page.north west]
  \draw [fill,GBV] (0mm,-6.4mm) rectangle +(\paperwidth,-11mm);
  \tikz[page position=current page.north east]
  \node [anchor=east,text=white] at (-25mm,-11.9mm) { \upshape\textsf{\@title} };
}

% Seitenzahlen in Weiß
\renewcommand{\pnumfont}{\color{white}\upshape}

\cfoot{% Fußzeile: Linie
  \tikz[page position=current page.south west]
  \draw[draw=GBV,line width=0.5pt] (0,1cm) to +(\paperwidth,0);
}
\refoot{% Fußzeile: Seitezahl links
  \tikz[page position=current page.south west]
  \node[anchor=south west,fill=GBV,minimum height=0.25cm] at (2.5cm,1cm) {\pagemark};
}
\lofoot{% Fußzeile: Seitezahl rechts
  \tikz[page position=current page.south east]
  \node[anchor=south east,fill=GBV,minimum height=0.25cm] at (-2.5cm,1cm) {\pagemark};
}

% Inhaltsverzeichnis
%\usepackage{setspace} 
%\onehalfspacing 
%\BeforeStartingTOC[toc]{\singlespacing}

\let\oldtoc\tableofcontents
\renewcommand*\tableofcontents{}

% Titelseite
%\newcommand\toconpage{\tableofcontents}

\renewcommand{\maketitle}{
  \begin{titlepage}
    % Linie oben
    \tikz[page position=current page.north west]
    \draw [fill,GBV] (0mm,-6.4mm) rectangle +(\paperwidth,-11mm);
    % Linie unten
    \tikz[page position=current page.south west]
    \draw[draw=GBV,line width=0.5pt] (0,1cm) to +(\paperwidth,0);

    \begin{flushright}
      \color{GBV} \bfseries

      {\Huge\@title} \\

      \vspace{5mm}

      \Large \@date % TODO: version
       \\ 
        \vspace{5mm}
        \@author
    \end{flushright}

    % Put ToC on the cover
    \hypersetup{linkcolor=black}
    \vfill
    \setcounter{tocdepth}{3}

    \begingroup
    \let\clearpage\relax
    \oldtoc
    \endgroup

    \vfill

  \end{titlepage}
}

% Schönere verbatim-Umgebung
\usepackage{fancyvrb}
\RecustomVerbatimEnvironment{verbatim}{Verbatim}{%
    frame=leftline,rulecolor=\color{lightgray},
    framerule=0.5pt,xleftmargin=1em,framesep=1em}

% Link-Farben
\definecolor{GBVDark}{RGB}{0,51,153}
\hypersetup{citecolor=GBVDark,urlcolor=GBVDark,linkcolor=GBVDark}

\makeatother
